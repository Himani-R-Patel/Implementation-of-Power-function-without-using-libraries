\documentclass[a4paper, 11pt]{article}
\usepackage{comment} % enables the use of multi-line comments (\ifx \fi) 
\usepackage{fullpage} % changes the margin
\usepackage{hyperref}
\usepackage{amsmath}
\usepackage{environ}
\usepackage{tabto,enumitem}
\usepackage{lipsum}

\usepackage{tabto}
\usepackage{booktabs} % For formal tables

\usepackage[ruled]{algorithm2e} % For algorithms
\renewcommand{\algorithmcfname}{ALGORITHM}

\begin{document}

\noindent
\large\textbf{PROBLEM 1} \hfill \textbf{Himani Patel} \\
\normalsize SOEN 6011 \hfill \textbf{40071101} \\
Function 9 :  $f(x,y)= x^y$  \hfill Date: 05/07/2019 \\

\section{Introduction}

F9:  $f(x,y)= x^y$ is a power function where x is a base and y is a exponent or power. Here, x and y both are a real variable. This function is one of the most commonly used function in mathematics.

\section{Domain \& Co-Domain}

Lets define function from A to B, represented as  $ f: A \to B$ , where A is the domain and B is the co-domain of the Function.

\subsection{Domain}
\begin{itemize}
  \item it includes all the real numbers.
   To be specific,  For  $ x > 0$ ,  $y \in R$	
  \item $For$ $ x = 0$ , $ y \ge 0$.
  \item $For$ $ x < 0$ , $ y \in Q $.
 \end{itemize}
  
 \subsection{Co-Domain}
 \begin{itemize}
  \item  $For$ $ x > 0$ , range is $[0,\infty)$  where $ x \in R$  and  $y \in R$. 
  \item  $For$ $ x = 0$ and $y =0$, range is 1 and  $For$ $ x = 0$ and $y>0$, range is 0.
 \item  $For$ $ x < 0$ , range is $(-\infty,\infty)$  where $ x \in R$  and  $y \in Z$.
\end{itemize}

\section{Characteristics}

\begin{itemize}
  \item $\boldsymbol{Parity}$ : This function is neither even nor odd.

  \item $\boldsymbol{Periodicity}$ : This function is periodic in y with period  \[\frac{-2\pi}{\log{(x)}sgn(\log{(x)})}\]
  
  \item $ \textbf{Injectivity \& Surjectivity}$ : This function is not injective which means it is not one-to-one function but it is surjective which means it is onto function.
  
   \item $\boldsymbol{Commutativity}$:  This function is not commutative which means $x^y \ne y^x $ for $x\ne y$.
  
\end{itemize}

\begin{thebibliography}{9}
\bibitem{Wolframalpha}
Wolframalpha,\\
\url{https://www.wolframalpha.com/input/?i=x%5Ey}
\bibitem{TutorialsPoint}
TutorialsPoint,\\
\url{https://www.tutorialspoint.com/java/lang/math_pow.htm}
\end{thebibliography}

\end{document}
 