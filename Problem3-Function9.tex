\documentclass{article}
\usepackage{algorithm}
\usepackage{algpseudocode}
\usepackage{amsmath}
\usepackage{amssymb}

\begin{document}

\noindent
\large\textbf{PROBLEM 3} \hfill \textbf{Himani Patel} \\
\normalsize SOEN 6011 \hfill \textbf{40071101} \\
Function 9 :  $f(x,y)=x^y$  \hfill Date: 19/07/2019 \\

\section{Introduction}

\noindent There are various ways to implement power function such as using iterative method , Recursive method , Taylor's Series, and so on. \\
The Taylor series for the exponential function $e^x$ at a = 0 is represented as,\\

$\sum_{n=0}^{\infty} \frac{x^n}{n!}= \frac{x^0}{0!}+\frac{x^1}{1!} +\frac{x^2}{2!} +\frac{x^3}{3!}+ .....$\\

\noindent
as function is $f(x) = x^y$ is also represented as $(y \ln x)$ 
Hence, by substitute  $(y \ln x)$ as x in the above series,\\

$x = e^{y \ln x}$  \\

Therefore , the above series can also be represented as,\\

$e^x = e^x \ln a $ = $ 1 + \frac{(y\ln x) }{1!}+\frac{(y\ln x)^2}{2!} +\frac{(y\ln x)^3}{3!}+ ... $\\

\noindent
In order to find the value of the above series , we have to calculate the value of $(y\ln x)$.\\

\noindent
For that, we will use the Taylor's series for the natural logarithm which is represented as,\\

$\ln x = 2[(\frac{x-1}{x+1}) + \frac{1}{3}(\frac{x-1}{x+1})^3 + \frac{1}{5}(\frac{x-1}{x+1})^5 + ....]$ \\

In general, $\ln x = 2\sum_{k=1}^{\infty} \frac{1}{2k-1} (\frac{x-1}{x+1})^{2k-1}$ \\

\begin{algorithm}

\caption{Calculate Power Function}

\textbf{Require:}  value: For  $ x > 0$ ,  $y \in R$ ;	
   $For$ $ x = 0$ , $ y \ge 0$ ;
   $For$ $ x < 0$ , $ y \in Q $.
\textbf{Ensure:} $result =  x^y$

\begin{algorithmic}[1]


\Procedure {CalculateFactorial}{$number$}
   \State $factorial \leftarrow 1$
   \State  $factorial \leftarrow number * \Call{CalculateFactorial}{number-1} $
   \State \textbf{return} $factorial$\Comment{It returns the factorial of the givem number}
    \EndProcedure 
\Statex


\Procedure {CalculatePowerForInteger}{$x$ , $y$}
     \State $power \leftarrow 1$
    \For {$i \leftarrow 1, y$}
    \State $power \leftarrow power * x$
    \EndFor
    \State \textbf{return} $power$\Comment{It returns the x raised to the power of y }
    \EndProcedure
\Statex

\Procedure {CalculateNaturalLog}{$number$}
    \State $ans \leftarrow 0$
    \State $base \leftarrow (number-1) / (number+1)$
    \For {$i \leftarrow 1, 10$}
    \State $exponent \leftarrow (2 * i) -1$
    \State $ans \leftarrow ans + (1/exponent) * \Call{CalculatePowerForInteger}{base,exponent} $
    \EndFor
    \State \textbf{return} $2*ans$\Comment{It returns natural logarithm for the given number.}
    \EndProcedure
\Statex

\Procedure {CalculatePowerForReal}{$x$,$y$}
    \State $answer \leftarrow 0$
     \State $logValue \leftarrow \Call{CalculateNaturalLog}{x} $
       
    \For {$i \leftarrow 1,10$}
    \State $numerator \leftarrow \Call{CalculatePowerForInteger}{(y*logValue),i} $
    
    \State $denominator \leftarrow \Call{CalculateFactorial}{i} $
    
    \State $answer \leftarrow answer + (numerator/denominator)$
    \EndFor
    \State \textbf{return} $answer$\Comment{Returns the Final Answer of $x^y$}
    \EndProcedure
\Statex

\end{algorithmic}
\end{algorithm}



\begin{algorithm}

\caption{Calculate Power Function}

\textbf{Require:}  value: For  $ x > 0$ ,  $y \in R$ ;	
   $For$ $ x = 0$ , $ y \ge 0$ ;
   $For$ $ x < 0$ , $ y \in Q $.
\textbf{Ensure:} $result =  x^y$

\begin{algorithmic}[1]
 \State $nthRoot \leftarrow 1$
  \State $exponent \leftarrow 1 $ 
  \Statex
  
\Procedure {CalculateRational}{$x$ , $y$}
   \State $SValue \leftarrow (String)number $
   \State $SLength \leftarrow length\hspace{0.15cm}  of\hspace{0.15cm}  SValue $
   \State $index \leftarrow index\hspace{0.15cm} of\hspace{0.15cm}  decimal\hspace{0.15cm}  point $
   \State $digits \leftarrow SLength -1 -index  $
   
   \For {$i \leftarrow 0, digits$}
   \State $d \leftarrow d * 10$
   \State $r \leftarrow nthRoot * 10$
   \EndFor
    \State $exponent \leftarrow d$
    \State $nthRoot \leftarrow r$
\Statex  


\noindent\Procedure {CalculatePower}{$x$ , $y$}
    
    \State $answer \leftarrow 1 $
    \State $base \leftarrow 0 $
    
   \If{$n \% 1 \leftarrow 0$}
    \State $base \leftarrow x $
     \State $exponent \leftarrow y $
   
   \Else
   
    \State\Call{CalculateRational}{y}
    \State $base \leftarrow nthRoot \hspace{0.15cm} of \hspace{0.15cm} x$
    
    \EndIf
    
     \If{$(exponent < 0)$}
     \State $exponent \leftarrow absoluteValue(exponent) $
     \EndIf
     
      \For {$i \leftarrow 0, exponent$}
       \State $answer \leftarrow answer* base$
       \EndFor
      
    \If{$y < 0$}
    \State $answer \leftarrow 1/answer $
    \EndIf
    
    \State \textbf{return} $answer$\Comment{Returns the Final Answer of $x^y$}
    
\end{algorithmic}

\end{algorithm}

\newpage

\section{Advantages & Disadvantages for Algorithm 1}

\subsection{Advantages}

\begin{itemize}
    \item It gives the incredibly accurate answer depending on the number of iterations used in the taylor series.
    \item Taylor series method is well-established series.Hence, it is easy to understand and simple to implement.
   
    
\end{itemize}

\subsection{Disadvantages}
\begin{itemize}
    \item Time to execute is depend on the number of iterations used.
    \item Truncation error tends to grow rapidly away from expansion point.
\end{itemize}


\section{Advantages and Disadvantages for Algorithm2}
\subsection{Advantages}
\begin{itemize}
    \item It is quite faster compared to the second approach.
    
    
\end{itemize}

\subsection{Disadvantages}
\begin{itemize}
    \item Finding of nth root of value is quite difficult.
    \item Multiple conditions need to consider.
    \item Need to handle the case separately when the exponent is the fractional number.
    
    
\end{itemize}




\end{document}